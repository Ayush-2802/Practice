\section{Advanced Time Series Analysis Approaches}
\subsection{Beyond Traditional ARIMA}
While our base implementation utilized ARIMA modeling, we explored several advanced time series methodologies to enhance forecasting accuracy:

\subsubsection{Multivariate Extensions}
Vector Autoregression (VAR) models significantly improved performance by capturing interdependencies between parameters. Our implementation revealed strong cross-correlations between tension and pressure parameters (Pearson's r = 0.83), enabling more accurate predictions through joint modeling. VAR models reduced forecasting error by 27\% compared to independent ARIMA models.

\begin{figure}[t]
\centering
\includegraphics[width=0.9\linewidth]{figures/time-series-models.pdf}
\caption{Comparative performance of different time series forecasting approaches on slitting machine sensor data, showing prediction error rates and computation time requirements.}
\label{fig:time_series_models}
\end{figure}

\subsubsection{Non-linear Time Series Models}
For capturing complex patterns in slitting machine behavior, we implemented:
\begin{itemize}
    \item Threshold Autoregressive (TAR) models that account for regime-switching behavior when materials change
    \item GARCH models specifically for tension variance forecasting, which proved particularly effective for detecting blade dulling effects
    \item State Space Models using Unobserved Components Models (UCM) to separate trend, seasonal, and cyclical components in pressure measurements
\end{itemize}

\subsubsection{Advanced Decomposition Methods}
We applied Seasonal-Trend decomposition using LOESS (STL) to separate production data into:
\begin{itemize}
    \item Long-term trends reflecting gradual component wear
    \item Periodic patterns corresponding to maintenance cycles
    \item Residual variations indicating potential anomalies
\end{itemize}

\begin{figure}[t]
\centering
\includegraphics[width=0.85\linewidth]{figures/time-series-decomposition.pdf}
\caption{Time series decomposition of tension sensor data showing trend, seasonal, and residual components with anomaly detection capabilities.}
\label{fig:decomposition}
\end{figure}

This decomposition approach improved early detection of subtle degradation patterns by 34\% compared to direct time series analysis.

\subsection{Deep Learning for Time Series}
Our research explored several neural network architectures specifically designed for temporal industrial data:

\subsubsection{Temporal Convolutional Networks}
Temporal Convolutional Networks (TCNs) outperformed traditional RNN-based approaches, achieving:
\begin{itemize}
    \item 23\% lower RMSE for 30-minute forecasts
    \item 41\% reduction in training time
    \item Greater robustness to noise and missing values
\end{itemize}

The dilated causal convolutions in TCNs proved particularly effective for capturing multi-scale patterns in slitting machine operation.

\subsubsection{Transformer-based Models}
We implemented a specialized transformer architecture for capturing long-range dependencies in maintenance data:
\begin{itemize}
    \item Self-attention mechanisms effectively identified correlations between events separated by hours or days
    \item Multi-head attention successfully modeled interactions between different parameter types
    \item Positional encodings adapted for irregular time intervals between measurements
\end{itemize}

Our transformer implementation achieved 18\% higher accuracy for predicting failures beyond 30 minutes compared to ARIMA models.

\subsubsection{Probabilistic Forecasting}
To better quantify uncertainty in predictions, we implemented:
\begin{itemize}
    \item DeepAR for generating prediction intervals, providing maintenance teams with confidence levels for each forecast
    \item Gaussian Process regression for capturing uncertainty in regions with sparse historical data
    \item Bayesian Neural Networks that produce probability distributions rather than point estimates
\end{itemize}

This probabilistic approach reduced false alarms by 32\% while maintaining sensitivity to actual failure precursors.