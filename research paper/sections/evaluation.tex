\section{Evaluation Beyond Standard Metrics}
\subsection{Economic Impact Assessment}
We developed comprehensive frameworks for quantifying business value:

\subsubsection{Total Cost of Ownership Analysis}
Our TCO model incorporated:
\begin{itemize}
    \item Direct maintenance costs (labor, parts, tools)
    \item Production opportunity costs from downtime
    \item Quality-related costs from defective production
    \item System implementation and operation costs
    \item Training and change management costs
\end{itemize}

\begin{figure}[t]
\centering
\includegraphics[width=0.9\linewidth]{figures/economic-impact.pdf}
\caption{Economic impact analysis showing cost savings and ROI breakdown by category over the 24-month deployment period.}
\label{fig:economic_impact}
\end{figure}

Analysis demonstrated a 3.2:1 return on investment with full payback achieved within 9.5 months of deployment.

\subsubsection{Risk-Adjusted Value Assessment}
To account for uncertainty, we implemented:
\begin{itemize}
    \item Monte Carlo simulation modeling different failure scenarios and their probabilities
    \item Value-at-risk calculations for worst-case scenarios
    \item Scenario analysis covering various production volumes and product mixes
    \item Sensitivity analysis identifying key variables affecting economic outcomes
\end{itemize}

This analysis revealed that the system provided a positive ROI with 97\% confidence even under pessimistic assumptions.

\subsubsection{OEE Integration}
We developed metrics linking predictive maintenance directly to Overall Equipment Effectiveness:
\begin{itemize}
    \item Availability impact through reduced unplanned downtime (increased by 7.3\%)
    \item Performance impact through optimized machine operation (increased by 4.1\%)
    \item Quality impact through reduced defect rates (increased by 5.8\%)
\end{itemize}

The combined OEE improvement of 13.2\% translated to approximately \$720,000 annual value for a single production line.

\subsection{Human Factors in Predictive Maintenance}
Our implementation paid careful attention to human-system interaction:

\subsubsection{Trust Calibration}
To ensure appropriate reliance on system recommendations, we implemented:
\begin{itemize}
    \item Confidence visualization proportional to prediction certainty
    \item Historical performance tracking for specific failure modes
    \item Explicit enumeration of factors contributing to uncertainty
    \item Progressive disclosure of detailed rationale on demand
\end{itemize}

This approach reduced both over-reliance and under-reliance on system recommendations, as measured through structured user studies.

\subsubsection{Alert Optimization}
To mitigate alert fatigue, we developed:
\begin{itemize}
    \item Context-aware alert thresholds that adapt to staff availability
    \item Priority-based notification routing based on urgency and expertise
    \item Consolidated alerts that group related issues
    \item Explicit acknowledgment tracking and escalation protocols
\end{itemize}

\begin{figure}[t]
\centering
\includegraphics[width=0.85\linewidth]{figures/alert-optimization.pdf}
\caption{Alert optimization framework showing the reduction in notification volume and increase in relevant actionable alerts after implementation.}
\label{fig:alert_optimization}
\end{figure}

These mechanisms reduced ignored alerts by 78\% while ensuring critical issues received timely attention.

\subsubsection{Knowledge Management Integration}
Our system incorporated formal knowledge management capabilities:
\begin{itemize}
    \item Capture of expert technician rationale during maintenance events
    \item Case-based reasoning to retrieve similar historical incidents
    \item Collaborative annotation of unusual failure modes
    \item Integration of manufacturer documentation and technical bulletins
\end{itemize}

This knowledge integration accelerated training of new maintenance personnel while preserving institutional knowledge despite staff turnover.