\section{Conclusion}
This research demonstrated the effectiveness of a hybrid predictive maintenance approach combining ARIMA forecasting with supervised learning for slitting machines in manufacturing environments. Our approach achieved 89\% accuracy in predicting potential failures up to 45 minutes before occurrence, significantly outperforming baseline methods and commercial solutions.

\begin{figure}[t]
\centering
\includegraphics[width=0.85\linewidth]{figures/future-work.pdf}
\caption{Summary of key contributions and future research directions for predictive maintenance of industrial slitting machines.}
\label{fig:conclusion}
\end{figure}

The enhancements to our original approach—particularly in sensor placement optimization, edge computing implementation, advanced time series modeling, and explainable AI integration—collectively produced a system that delivers substantial business value through reduced downtime, lower maintenance costs, and improved product quality.

Our implementation represents a significant advancement in predictive maintenance for slitting machines, a critical yet under-researched area in manufacturing. The comprehensive approach addressing technical, operational, and human factors provides a blueprint for effective predictive maintenance deployment in similar industrial contexts.

As industrial IoT adoption continues to accelerate, predictive maintenance approaches like the one presented in this research will play an increasingly important role in optimizing manufacturing operations, reducing costs, and improving equipment reliability.