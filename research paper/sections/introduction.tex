\section{Introduction}
Manufacturing industries face significant challenges from unplanned machine downtime, which can cost up to \$260,000 per hour according to recent industry reports. Traditional maintenance approaches—reactive and preventive—fail to optimize maintenance scheduling, often leading to either unnecessary servicing or unexpected failures. The emergence of Industry 4.0 technologies, particularly the Industrial Internet of Things (IIoT), presents new opportunities for more efficient maintenance strategies.

\begin{figure}[t]
\centering
\includegraphics[width=0.85\linewidth]{figures/system-overview.pdf}
\caption{Overview of the proposed predictive maintenance system architecture showing data flow from IoT sensors through processing layers to maintenance decision support.}
\label{fig:system_overview}
\end{figure}

Predictive maintenance (PdM) represents a paradigm shift from these traditional approaches, employing data-driven methods to predict when equipment will fail and enabling just-in-time maintenance interventions. By monitoring machine health in real-time and applying machine learning algorithms to sensor data, manufacturers can detect early warning signs of degradation, optimize repair scheduling, and extend equipment lifespan.

In this research, we focus on slitting machines—critical equipment in the packaging industry used to convert large master rolls into smaller rolls of specific dimensions. These machines operate continuously and any unexpected stoppage results in significant production losses and quality issues. Despite their importance, slitting machines have received limited attention in predictive maintenance research literature.

Our work addresses this gap by developing and implementing a hybrid predictive maintenance system specifically designed for slitting machines. We combine time series forecasting through ARIMA modeling with supervised learning classification to predict potential failures before they occur, offering maintenance teams sufficient time to take corrective actions.

The primary contributions of this research include:
\begin{enumerate}
    \item A comprehensive data collection and preprocessing pipeline specifically designed for slitting machine sensor data
    \item A novel hybrid model combining ARIMA forecasting with supervised classification for failure prediction
    \item Experimental validation on real-world industrial data from a packaging film slitting machine
    \item Quantitative evaluation demonstrating the effectiveness of our approach for quality issue prediction
\end{enumerate}