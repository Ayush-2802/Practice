\section{Related Work}
Recent advances in sensor technology, IoT infrastructure, and machine learning algorithms have accelerated research in predictive maintenance for industrial applications. This section presents key related work relevant to our research.

\subsection{IoT-Based Predictive Maintenance Systems}
Recent implementations of IoT-based predictive maintenance have demonstrated significant benefits across various industries. Martinez and Lee \cite{martinez2024} developed an explainable AI framework for predictive maintenance that achieved 92\% fault classification accuracy while providing interpretable decision models for maintenance planning. Their approach employed deep learning combined with explainability techniques to make predictions more transparent to maintenance personnel.

\begin{figure}[t]
\centering
\includegraphics[width=0.9\linewidth]{figures/related-work-comparison.pdf}
\caption{Comparison of recent IoT-based predictive maintenance approaches highlighting accuracy and lead time performance across different industrial applications.}
\label{fig:related_work}
\end{figure}

Wang et al. \cite{wang2024} addressed security concerns in IoT maintenance systems by implementing a blockchain-based framework for secure data sharing. Their approach achieved 99.9\% data integrity, enabling trustworthy collaboration between maintenance stakeholders.

Zhang et al. \cite{zhang2023} focused on remaining useful life (RUL) prediction using deep learning approaches with IoT sensor networks, achieving 87\% accuracy in predicting component lifespans. Their work particularly emphasized the use of LSTM networks for capturing temporal dependencies in sensor data.

\subsection{Machine Learning for Failure Prediction}
Machine learning approaches for failure prediction have evolved from simple statistical methods to sophisticated deep learning architectures. Kim and Park \cite{kim2023} developed an edge computing-based framework for real-time predictive maintenance that reduced response time by 65\%, enabling faster decision-making in critical manufacturing processes.

Johnson et al. \cite{johnson2023} explored transfer learning approaches for fault detection across different machine types, achieving 84\% accuracy when applying knowledge from one domain to another. This is particularly relevant for manufacturing environments with diverse equipment.

Liu et al. \cite{liu2023} addressed privacy concerns by implementing a federated learning architecture that achieved 89\% accuracy without sharing sensitive production data between facilities. Their approach enables collaborative model training while preserving data confidentiality.

\subsection{Time Series Analysis in Predictive Maintenance}
Time series analysis techniques have proven effective for capturing temporal patterns in machine behavior. Chen et al. \cite{chen2023} integrated digital twin technology with predictive maintenance using IoT and machine learning, achieving 91\% failure prediction accuracy by simulating machine behavior under various conditions.

Williams et al. \cite{williams2023} demonstrated that multi-sensor data fusion enhanced fault diagnosis accuracy to 94\%, showing that combining different sensor modalities improves detection performance. Their approach used deep learning to integrate vibration, temperature, and acoustic data.

Brown et al. \cite{brown2024} combined physics-based models with neural networks to achieve 88\% prediction accuracy, demonstrating the value of incorporating domain knowledge into machine learning approaches.