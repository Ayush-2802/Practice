\section{Hybrid Model Architecture Enhancements}
\subsection{Model Integration Strategies}
Our research extended beyond simple sequential pipelines to develop sophisticated integration approaches:

\subsubsection{Multi-task Learning Framework}
We developed a joint optimization framework that simultaneously addresses:
\begin{itemize}
    \item Parameter forecasting (regression task)
    \item Anomaly detection (binary classification)
    \item Failure type identification (multi-class classification)
    \item Remaining useful life estimation (regression task)
\end{itemize}

\begin{figure}[t]
\centering
\includegraphics[width=0.9\linewidth]{figures/hybrid-model-architecture.pdf}
\caption{Detailed architecture of our hybrid predictive maintenance model showing the integration of time series forecasting, feature extraction, and classification components.}
\label{fig:hybrid_architecture}
\end{figure}

This multi-task approach leveraged shared representations across tasks, improving overall performance by 17\% compared to individual specialized models.

\subsubsection{Hierarchical Ensemble Architecture}
Our enhanced architecture implemented a three-level hierarchical ensemble:
\begin{enumerate}
    \item Base-level models operating on individual sensor streams
    \item Mid-level models integrating related parameter groups (mechanical, electrical, material-related)
    \item Meta-level models combining predictions using stacking with gradient boosting
\end{enumerate}

This hierarchical approach improved overall accuracy from 89\% to 93\% while providing more robust performance across different failure modes.

\subsubsection{Dynamic Model Selection}
We implemented a context-aware model selection system that:
\begin{itemize}
    \item Adapts to changing operating conditions by selecting appropriate model configurations
    \item Employs Bayesian optimization to continuously evaluate model performance
    \item Incorporates domain-specific rules during critical production phases
\end{itemize}

This adaptive approach proved particularly valuable during material changeovers, where model performance typically degraded with fixed architectures.

\subsection{Explainable AI Integration}
To enhance trust and adoption among maintenance personnel, we integrated several explainability techniques:

\subsubsection{Local Interpretable Model-agnostic Explanations}
We implemented LIME and SHAP approaches to explain individual predictions:
\begin{itemize}
    \item Interactive visualizations showing contribution of each feature to specific predictions
    \item Confidence intervals for feature importance estimates
    \item Comparative analysis against historical similar cases
\end{itemize}

\begin{figure}[t]
\centering
\includegraphics[width=0.85\linewidth]{figures/explainable-ai.pdf}
\caption{SHAP value visualization for a failure prediction case, showing the contribution of key features to the model's decision and helping maintenance personnel interpret the results.}
\label{fig:explainable_ai}
\end{figure}

Maintenance technicians reported 68\% higher confidence in system recommendations when provided with these explanations.

\subsubsection{Counterfactual Explanations}
We developed a novel approach for generating actionable maintenance insights:
\begin{itemize}
    \item "What-if" scenarios showing how parameter adjustments would affect failure probability
    \item Minimum change required to reduce failure risk below critical thresholds
    \item Interactive tools allowing maintenance personnel to explore intervention options
\end{itemize}

This approach reduced average decision time for maintenance interventions from 27 minutes to 8 minutes.

\subsubsection{Visualization Techniques}
We implemented specialized visualization methods for temporal pattern recognition:
\begin{itemize}
    \item Phase portraits highlighting system trajectory toward failure states
    \item Shapelet-based visualizations identifying discriminative subsequences
    \item Attention maps showing which time periods most strongly influence predictions
\end{itemize}

These visualizations proved particularly effective for training new maintenance personnel, reducing onboarding time by 42\%.