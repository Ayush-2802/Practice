\section{IoT-Based Predictive Maintenance Infrastructure}
\subsection{Sensor Selection and Placement Strategy}
For slitting machines, strategic sensor placement significantly impacts prediction accuracy. Our research explored several key considerations for sensor deployment:

\subsubsection{Optimal Sensor Positioning}
We employed Design of Experiments (DoE) methodologies to identify critical measurement points. This systematic approach revealed that placing vibration sensors at specific points along the slitting blades provided early detection of misalignment issues. Specifically, we found that sensors positioned at 30-degree intervals around critical bearings captured 95\% of failure precursors, while conventional equidistant placement captured only 78\%.

\begin{figure}[t]
\centering
\includegraphics[width=0.85\linewidth]{figures/sensor-placement.pdf}
\caption{Optimal sensor placement strategy for slitting machine monitoring. Red circles indicate vibration sensors, blue squares represent temperature sensors, and green triangles show acoustic sensors.}
\label{fig:sensor_placement}
\end{figure}

\subsubsection{Sensor Density Optimization}
Our experiments revealed a non-linear relationship between sensor density and predictive performance. Using incremental sensor deployment, we determined that 8 sensors per slitting arm represented the optimal balance between data richness and cost-effectiveness. Additional sensors beyond this threshold provided diminishing returns, improving prediction accuracy by less than 2\% while increasing implementation costs by 40\%.

\subsubsection{Advanced Sensing Technologies}
Beyond conventional sensors, we implemented:
\begin{itemize}
    \item Piezoelectric sensors capable of detecting high-frequency vibrations (up to 20kHz), which revealed subtle blade resonance patterns 30 minutes before visible quality degradation
    \item Thermal imaging cameras providing non-contact temperature monitoring, which identified hotspots related to bearing failures with 92\% accuracy
    \item Acoustic sensors detecting ultrasonic signatures of internal component wear, particularly effective for detecting lubricant degradation
\end{itemize}

\subsection{Edge Computing Architecture}
Our implementation leveraged advanced edge computing capabilities to optimize data processing:

\subsubsection{Distributed Processing Framework}
We developed a hierarchical edge computing architecture with three tiers:
\begin{enumerate}
    \item Sensor-level preprocessing using embedded microcontrollers that implement data filtering and feature extraction
    \item Machine-level processing using industrial PCs that perform preliminary anomaly detection
    \item Plant-level aggregation servers that coordinate multi-machine analytics
\end{enumerate}

\begin{figure}[t]
\centering
\includegraphics[width=0.9\linewidth]{figures/edge-architecture.pdf}
\caption{Three-tier edge computing architecture showing data flow and processing responsibilities across sensor, machine, and plant levels.}
\label{fig:edge_architecture}
\end{figure}

This distributed approach reduced network bandwidth requirements by 76\% while decreasing cloud processing costs by 65\%.

\subsubsection{Time-Series Optimization Techniques}
To handle the high-velocity data streams (over 3,600 data points per minute), we implemented:
\begin{itemize}
    \item Piecewise Aggregate Approximation (PAA) for dimensionality reduction, preserving 97\% of signal information while reducing storage requirements by 80\%
    \item Adaptive sampling rates that automatically increased frequency during detected anomalies
    \item Symbolic Aggregate approXimation (SAX) for efficient pattern matching against known failure signatures
\end{itemize}

\subsubsection{Hardware Acceleration}
For real-time processing requirements, we deployed:
\begin{itemize}
    \item FPGA-based signal processing for high-frequency vibration analysis, reducing latency from 250ms to 12ms
    \item GPU acceleration for neural network inference at the edge, enabling complex model execution without cloud connectivity
    \item Custom ASICs for specific time-critical operations, particularly for tension control feedback loops
\end{itemize}